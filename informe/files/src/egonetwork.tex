% === CONTEXTO === %

\vspace{2em}
\subsection{Contexto} Una \textit{ego-network} \cite{Leskovec} es una red compuesta por las amistades que existen entre los amigos de un individuo, el `ego'. Estas redes son centrales para aplicaciones como Facebook, Google+ o Twitter. 

En particular, dada una red `ego', resulta de interés poder identificar los circulos sociales ---conjuntos, disjuntos y anidados--- a los que pertenece un usuario. Leskovec \cite{Leskovec} propone un método de aprendizaje no supervisado para lograr inferirlos, que se nutre de la siguiente información: un grafo \textbf{E} (la red) ---donde se espera que exista una correlación fuerte entre un círculo y la densidad de conexiones entre los nodos que lo componen--- y un conjunto de atributos \textbf{C}, para cada nodo ---donde se espera una correlación entre un círculo y la similaridad de los atributos de los nodos que lo componen---.

\vspace{1em}
En este análisis estimaremos \textbf{E}, una red `ego' proveniente de Facebook, por medio de la construcción de una matriz de similaridad que utilice los atributos definidos en \textbf{C}. También, buscaremos reducir su dimensionalidad por medio del análisis de componentes principales.




% === SIMILARIDAD @ ATRIBUTOS === %

\vspace{2em}
\subsection{Matriz de similaridad}




% === COMPARACION === %

\vspace{2em}
\subsection{Comparación con la red original}




% === OPTIMIZACION === %

\vspace{2em}
\subsection{Optimización}




% === PCA === %

\vspace{2em}
\subsection{PCA}
