% === CONTEXTO === %

\vspace{1em}
\subsection{Contexto}

La red del \textit{Club de Karate} es parte de una investigación antropológica \TODO{citar} que estudió las relaciones `políticas' entre los miembros de un club universitario. La misma se realizó durante el desarrollo de un conflicto que terminó por dividir al grupo. La red busca modelar el flujo de información entre sus integrantes por medio de la tripla $(\mathbf{V},\ \mathbf{E},\ \mathbf{C})$, donde $\mathbf{V}$ denota el conjunto de individuos, $\mathbf{E}$ refiere a un grafo no direccionado cuyos ejes representan los vínculos entre los miembros, y $\mathbf{C}$ define la fuerza de estas relaciones ---lo que se podría pensar como ponderaciones sobre los ejes de $\mathbf{E}$---. 

La investigación tuvo como eje central demostrar la capacidad del modelo para predecir la división del grupo por medio del algoritmo de \textit{labeling} de `flujo máximo - corte mínimo' de Ford y Fulkelson \TODO{citar}.

\vspace{1em}
En este análisis utilizaremos sólo la representación matricial de $\mathbf{E}$ para evaluar la importancia de los distintos miembros en la red, y la matriz laplaciana asociada para evaluar el uso de autovectores como predictores de la división del grupo. 




% === CENTRALIDAD DE AV === %

\vspace{2em}
\subsection{Centralidad de Autovector} La centralidad de autovector es una medida que se utiliza en el análisis de redes para medir la `importancia' de los nodos que componen una red, relativa a la importancia de sus conexiones. Dada una matriz de conectividad \textbf{W}, se define:

\vspace{1em}
\begin{equation} \label{conectividad}
    \lambda x = \mathbf{W} x
\end{equation}

\vspace{1em}
\noindent donde $\lambda$ es el autovalor en módulo máximo de \textbf{W}.

\vspace{1em}
Intuitivamente, se puede pensar que la importancia de cada nodo es proporcional a la suma de las importancias de sus vecinos \TODO{citar}. Se puede demostrar \TODO{citar} que, dado las características de esta matriz, el autovector asociado siempre será positivo en coordenadas.

\vspace{1em}
En tanto la red de \textit{Club de Karate}, podemos ver que la aplicación del metodo de la potencia con deflación sobre la matriz de conectividad asociada al grafo $\mathbf{E}$ resulta en el siguiente autovector asociado al autovalor en módulo máximo:

\vspace{1em}
\TODO{presentar vector.}

\vspace{1em}
Vemos que el nodo `1' y el nodo `34' son los más `centrales'. Esto no es casualidad, la red del \textit{Club de Karate} está armada para tener a las dos figuras centrales del conflicto en cada extremo ---el instructor de karate y el presidente del club---, para satisfacer la especificación del algoritmo de `labeling' que utiliza.




% === AV DE LA MATRIZ LAPLACIANA === %

\vspace{2em}
\subsection{Autovectores de la matriz laplaciana} La matriz laplaciana $\mathbf{L} = \mathbf{D} - \mathbf{W}$ ---donde \textbf{D} es una matriz diagonal con elementos $d_{ii} = \sum_j w_{ij}$ y \textbf{W} es una matriz de conectividad--- sirve para medir distintas propiedades en una red. En particular, el mínimo autovalor en módulo no nulo ---llamado de conectividad algebráica, o valor de Fiedler--- permite establecer un criterio sobre el que particionar la red en dos. El autovector asociado a este autovalor designará la pertenencia de un nodo a una u otra partición acorde a su signo. 

\vspace{1em}
Procedemos a analizar qué autovector de la matriz laplaciana asociada a $\mathbf{E}$ permite predecir mejor la división que ocurrió en el \textit{Club de Karate}. Para ello, medimos el valor absoluto de la correlación entre cada autovector y un vector que indica la división que ocurrió en el grupo.

\vspace{1em}
\TODO{presentar solución.}
