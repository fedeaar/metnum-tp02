La descomposición de matrices en autovectores y autovalores aparece en una variedad de aplicaciones donde importa caracterizar el comportamiento de un sistema: en el reconocimiento de imágenes, en el análisis de estabilidad de cuerpos rotantes, en el análisis de riesgo de mercado y en el análisis de redes ---por nombrar algunas---. Desde un punto de vista geométrico, se puede considerar a los autovectores como los `ejes' de una transformación lineal, en tanto representan una dirección invariante a la transformación, y a los autovalores como los factores por los que esas direcciones se comprimen, estiran o invierten. 

\vspace{1em}
En este trabajo propondremos una implementación en C++ de un método para el cálculo de autovalores reales, no nulos y en módulo dominantes, y sus respectivos autovectores, en matrices cuadradas. Para algunas matrices particulares, como pueden ser las matrices simétricas definidas positivas, este método nos permitirá obtener todos sus autovalores y autovectores asociados. El mismo se conoce como \textit{el método de la potencia con deflación}. 

A su vez, presentaremos dos aplicaciones concretas de los autovalores y autovectores en el análisis de redes: la medición de centralidad de autovector y corte mínimo en la red del \textit{`Club de Karate'} \cite{Zachary}, y la estimación de una \textit{ego-network} \cite{Leskovec} de Facebook, por medio de la construcción de una matriz de similaridad.  

\vspace{1em}
\noindent Palabras clave: \textit{método de la potencia, deflación de Hotelling, centralidad de autovector, conectividad algebráica, análisis de componentes principales.}
